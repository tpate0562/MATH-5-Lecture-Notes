 %!TeX root = MATH5 Lecture Notes.tex
\documentclass{article}
\usepackage[dvipsnames, svgnames, x11names]{xcolor}
\usepackage{tikz}
\usepackage{pgfplots}
\usepackage{pgfplotstable}
\usepackage{setspace}
\usepackage{units}
\usepackage{booktabs}
\usepackage{amssymb}
\usepackage{graphicx}
\usepackage{multirow}
\usepackage{amsopn}
\usepackage{bbding}
\usepackage{amsmath}
\usepackage{hyperref}
\usepackage{cancel}
\usepackage{gensymb}
\usepackage[margin = 1in]{geometry}
\AtBeginEnvironment{document}{\everymath{\displaystyle}}
\newtheorem{theorem}{1.4}
\title{MATH 5 Lecture Notes}
\date{Tuesday, 14 January, 2025}
\author{Tejas Patel}
\begin{document}
\maketitle
\tableofcontents
\pagebreak
\section{Chapter 1}
\subsection{Systems of Equations}
$
\left[\begin{array}{cc|c}
a & b & c \\
c & d & e
\end{array}\right]
$
Linear Equation Example:\\
$
a_{11}x_1 +a_{12}x_2 +...+ a_{1n}x_n =b_1\\
a_{21}x_1 +a_{22}x_2 +...+ a_{2n}x_n =b_2\\
a_{ij} = $ are coefficients\\ $
x_i = $ variables = unknowns\\ $
b_i =$ constants\\ 
The set $S_1, ... S_n$ is a solution if  $a_iS_1 +...+a_{in}S_n=b_i$ is True = consistent for all $i$
\\ \textbf{3 possible outcomes}\\
One Solution \\ No Solutions \\ Infinitely many Solutions
\\ \textbf{Row Operations}
\\Interchange Rows \\ Row Multiplication \\ Row Addition: Add a constant multiple of one row to another row
\\ Goal: $
\left[\begin{array}{cc|c}
1 & 0 & S_1 \\
0 & 1 & S_2
\end{array}\right]
$
\subsection{Row Reductions and Echelon Form}
\textbf{Reduced Row Echelon Form}
\begin{enumerate}
    \item Nonzero rows above zero rows
    \item Each leading entry in a row is to the right of the pivot in row above
    \item All pivots have zero above, below, \& left of the pivot
    \item All pivots must be 1
\end{enumerate}
\textbf{Theorem 1}: The RREF[A] form is unique, does not imply there is a unique solution
\subsubsection*{Ex. 1 Not RREF, just Row Echelon Form}
$
\left[\begin{array}{cccc|c}
3 & 2 & 3 & 0 & 6 \\
0 & 0 & 1 & 1 & 7 \\
0 & 0 & 0 & 1 & 5 \\
0 & 0 & 0 & 0 & 0 
\end{array}\right]
$ Zeroes below and left of each pivot. A pivot is the first nonzero entry in each row\\[0.05in] 
Basic variable corresponds to a column with a pivot. In this case $x_1, x_3, x_4$ are the basic variables
\\Free variable means there is no pivot in that column. In this case $x_2 = t$ to give it a parameter\\
$3x_1+2x_2+3x_3=6, x_2 = t \\ x_3+x_4=7, x_3=2 \\ x_4 = 5, x_4=5 \\ 3x_1+2t+3\cdot 2 = 6 \\ 3x_1+2t=0\\ 3x_1=-2t\\ x_1=-\frac{2}{3}t$ making the parametric solution to the system:
\boxed{(-\frac{2}{3}t, t, 2, 5)} or \boxed{(-\frac{2}{3}x_2, x_2, 2, 5)}

\subsubsection*{Ex. 2 Not RREF, just Row Echelon Form}
$\left[\begin{array}{cccc|c}
1 & 2 & 0 & 4 & 2 \\
0 & 0 & 1 & 3 & 6 \\
0 & 0 & 0 & 0 & 1 \\
\end{array}\right]$ No Solution — system is not consistent in row 3\\[0.05in]
\subsubsection*{Ex. 3 Convert to RREF}
$
\left[\begin{array}{cccc|c}
3 & 2 & 3 & 0 & 6 \\
0 & 0 & 1 & 1 & 7 \\
0 & 0 & 0 & 1 & 5
\end{array}\right] \rightarrow R_1 \mathrel{+}= -3R_2 \rightarrow
\left[\begin{array}{cccc|c}
3 & 2 & 0 & -3 & -15 \\
0 & 0 & 1 & 1 & 7 \\
0 & 0 & 0 & 1 & 5
\end{array}\right] \\ [0.1in]
\left[\begin{array}{cccc|c}
3 & 2 & 0 & -3 & -15 \\
0 & 0 & 1 & 1 & 7 \\
0 & 0 & 0 & 1 & 5
\end{array}\right]  \rightarrow R_1 \mathrel{+}= 3R_3 \rightarrow
\left[\begin{array}{cccc|c}
3 & 2 & 0 & 0 & 0 \\
0 & 0 & 1 & 0 & 2 \\
0 & 0 & 0 & 1 & 5
\end{array}\right] \\[0.1in]
\left[\begin{array}{cccc|c}
3 & 2 & 0 & 0 & 0 \\
0 & 0 & 1 & 0 & 2 \\
0 & 0 & 0 & 1 & 5
\end{array}\right]
 \rightarrow R_1 \mathrel{/}= 3 \rightarrow \left[\begin{array}{cccc|c}
1 & \frac{2}{3} & 0 & 0 & 0 \\
0 & 0 & 1 & 0 & 2 \\
0 & 0 & 0 & 1 & 5
\end{array}\right] 
$
\\Translate back to equations where $x_2=t$\\
$x_4 = 5, x_3=2, x_1+\frac{2}{3}x_2=0 \\ x_1 = -\frac{2}{3}t\\x_2=t\\x_3=2\\x_4=5 \; \boxed{\text{Solution:} (-\frac{2}{3}t, t, 2, 5) }$
\\If a solution exists and there is a free variable then there are infinitely many solutions 


\subsection{Vector Equations}
If ${\vec{a}_1...\vec{a}_j}$ spans $\mathbb{R}^n$ then any $\vec{b}$ in $\mathbb{R}^n$ can be written as a Linear Combination of ${\vec{a}_1...\vec{a}_j}$ so there exists ${{x}_1...{x}_j}$ so that ${x_1\vec{a}_1...x_j\vec{a}_j}=\vec{b}$ 
\\In math: ${\vec{a}_1...\vec{a}_j}$ spans $\mathbb{R}^n \Leftrightarrow \forall \vec{b} \epsilon \mathbb{R}^n \exists {{x}_1...{x}_j} \mathbb{R}$ so ${x_1\vec{a}_1...x_j\vec{a}_j}=\vec{b}$  
\\ \textbf{Theorem: }
If $\vec{v_1},..., \vec{v_p}, \vec{b}, \epsilon \mathbb{R}^n$ the following statements are equivalent:
\begin{enumerate}
    \item $\vec{b}$ is a linear combination of $\vec{v_1},...,\vec{v_p}$
    \item $\exists x_1,...,x_p \epsilon \mathbb{R}$ so $x_1\vec{v_1}+...x_p\vec{v_p}=b$
    \item $\vec{b} \epsilon$ spans {$\vec{v_1},..., \vec{v_p}$}
    \item \text{The linear \underbar{system} corresponding to augmented matrix with columns} $\left[\vec{v_1}...\vec{v_p} | \vec{b}\right]$ is consistent
    \item The reduced row echelon matrix of this augmented matrix does not contain a pivot in the last column $\left[0...0| 1 \right]$
\end{enumerate}
\subsection{The Matrix Equations Ax=b}
\textbf{Matrix Multiplication} $A\cdot \vec{b}= \left[\begin{array}{ccc}
    a_1&a_2&a_3
\end{array}\right]
\left[\begin{array}{c}
    b_1\\b_2\\b_3
\end{array}\right] = b_1\vec{a}_1 + b_2\vec{a}_2 + b_3\vec{a}_3$   \\Matrix A times a vector $\vec{b}$ 
\\[0.5in]\textbf{Compute the product or state that it is undefined}
$\left[\begin{array}{cc}
3&-7\\8&-4\\-4&1\end{array}\right]\left[\begin{array}{c}-4\\5
\end{array}\right] = \left[\begin{array}{c}-47\\-52\\21\end{array}\right]$
\textbf{Write the following system as a vector equation or matrix equation as indicated}
$x_1\left[\begin{array}{c}6\\0\end{array}\right]+ x_2\left[\begin{array}{c}2\\-1
\end{array}\right] + x_3 \left[\begin{array}{c}0\\3\end{array}\right] = \left[\begin{array}{c}4\\1\end{array}\right]$
\\\textbf{As a matrix Equation:}
$x_1\left[\begin{array}{ccc}6&2&0\\0&-1&3
\end{array}\right]\left[\begin{array}{c} x_1\\x_2\\x_3
\end{array}\right] = \left[\begin{array}{c}4\\1\end{array}\right]$
\\\textbf{Solve}
$\left[\begin{array}{ccc}1&3&-1\\2&-4&1\\-4&18&-5\end{array}\right]$ and $\left[\begin{array}{c}b_1\\b_2\\b_3\end{array}\right]$
\\$\left[\begin{array}{cccc}1&3&-1&b_1\\0&-10&3&-2b_1+b_2\\0&30&-9&4b_1+b_3\end{array}\right]$ 
\\[0.1in]$\left[\begin{array}{cccc}1&3&-1&b_1\\0&-10&3&-2b_1+b_2\\0&0&0&-2b_1+3b_2+b_3\end{array}\right]$ 
\\If $-2b_1+3b_2+b_3=0$ the system has infinitely many solutions. If its not on the plane then the system is inconsistent
\\\textbf{Theorem: }Let A be an $m\times n$ matrix. Then the following statements arae logically equivalent. That is, for a particular A, either they are all true or all false.
\begin{list}{a}
    \item For each \textbf{b} in $\mathbb{R}^m$, the equation $Ax=b$ has a solution
    \item Each B in $\mathbb{R}^m$ is a linear combination of the columns in $A$
    \item The Columns of $A$ span $\mathbb{R}^m$
    \item RREF of $A$ has a pivot position in every row
\end{list}
\subsection{Solution Sets for Linear Systems}
\subsection{Linear Independence}
\subsection{Linear Transformations}
\subsection{The Matrix of a Linear Transformation}
\section{Chapter 2}
\subsection{Matrix Operations}
\subsection{The Inverse of a Matrix}
\subsection{Characterizations of Invertible Matrices}
\section{Chapter 3}
\subsection{Introduction to Determinants}
\subsection{Properties of Determinants}
\subsection{Cramer's Rule and Linear Transformations}
\section{Chapter 4}
\subsection{Vector Spaces and Subspaces}
\subsection{Null Spaces and Column Spaces}
\subsection{Linear Independence}
\subsection{Coordinate Systems}
\subsection{Dimension and Rank}
\subsection{Change of Basis}
\section{Chapter 5}
\subsection{Eigenvectors and Eigenvalues}
\subsection{The Characteristic Equations}
\subsection{Diagonalization}
\subsection{Eigenvectors and Linear Transformations}
\subsection{Complex Eigenvalues}
\subsection{Discrete Dynamical Systems}
\subsection{Applications to Markov Chains}
\section{Chapter 6}
\subsection{Inner Product Spaces}
\subsection{Orthogonal Sets}
\subsection{Orthogonal Projections}
\subsection{The Gram Schmdit Process}
\subsection{Inner Product Spaces}
\pagebreak
\section{Example Problems with Solutions}
\textbf{1: Find the general solution to the homogenous system below}\\[0.1in]
$\left[\begin{array}{cccc|c}
2 & 0 & 1 & -1 & 0 \\
0 & 0 & -1 & 1 & 0 \\
1 & -1 & 0 & -11 & 0
\end{array}\right]$
\\[0.05in]Interchange 1 and 3 then $R_2 = R_1-2R_3$\\[0.05in]
$\left[\begin{array}{cccc|c}
1 & -1 & 0 & -1 & 0 \\
0 & 2 & 1 & 1 & 0 \\
0 & 0 & -1 & 1 & 0
\end{array}\right]$
\\[0.05in]Row 2 += Row 3 and Row 3 *= -1\\[0.05in]
$\left[\begin{array}{cccc|c}
1 & -1 & 0 & -1 & 0 \\
0 & 2 & 0 & 2 & 0 \\
0 & 0 & 1 & -1 & 0
\end{array}\right]$
\\[0.05in]Row 2 /= 2\\[0.05in]
$\left[\begin{array}{cccc|c}
1 & 0 & 0 & 0 & 0 \\
0 & 1 & 0 & 1 & 0 \\
0 & 0 & 1 & -1 & 0
\end{array}\right]$
Back into equations $x_1=0, x_2+x_4=0, x_3-x_4=0$ \\[0.05in]$x_4$ is a free variable since there is no pivot and the solution is $\boxed{0, -t, t, t}$\\
\textbf{2: Solve for h to make consistent \& find all solutions}\\[0.05in]
$\left[\begin{array}{cccc|c}
2 & -1 & 1 & -1 & 0 \\
0 & 0 & 1 & 2 & h \\
-2 & 1 & 6 & 15 & -21
\end{array}\right]$
\\[0.05in]Add first row to 3rd row\\[0.05in]
$\left[\begin{array}{cccc|c}
2 & -1 & 1 & -1 & 0 \\
0 & 0 & 1 & 2 & h \\
0 & 0 & 7 & 14 & -21
\end{array}\right]$
\\[0.05in]Scale row 3 by 1/7\\[0.05in]
$\left[\begin{array}{cccc|c}
2 & -1 & 1 & -1 & 0 \\
0 & 0 & 1 & 2 & h \\
0 & 0 & 1 & 2 & -3
\end{array}\right]$\\[0.05in]
A: Consistent if $h=-3$, inconsistent if $h\neq -3$\\
B: Find Solution when $h=-3$
$\left[\begin{array}{cccc|c}
1 & -1/2 & 0 & -3/2 & 3/2 \\
0 & 0 & 1 & 2 & -3 \\
0 & 0 & 0 & 0 & 0
\end{array}\right]$
\\[0.1in]Solution:
$\left[\begin{array}{c}
\frac{3}{2}-\frac{3}{2}t+\frac{1}{2}qs \\
s \\
-3-2t \\
t
\end{array}\right]$

\end{document}