 %!TeX root = MATH5 Lecture Notes.tex
\documentclass{article}
\usepackage[dvipsnames, svgnames, x11names]{xcolor}
\usepackage{tikz}
\usepackage{pgfplots}
\usepackage{pgfplotstable}
\usepackage{setspace}
\usepackage{units}
\usepackage{booktabs}
\usepackage{graphicx}
\usepackage{multirow}
\usepackage{amsopn}
\usepackage{bbding}
\usepackage{amsmath}
\usepackage{hyperref}
\usepackage{cancel}
\usepackage{gensymb}
\usepackage[margin = 1.2in]{geometry}
\AtBeginEnvironment{document}{\everymath{\displaystyle}}
\title{MATH 5 Lecture Notes}
\date{Tuesday, 14 January, 2025}
\author{Tejas Patel}
\begin{document}
\maketitle
\tableofcontents
\pagebreak
\section{Chapter 1}
\subsection{Systems of Equations}
\[
\left[\begin{array}{cc|c}
a & b & c \\
c & d & e
\end{array}\right]
\]
Linear Equation Example:\\
$
a_{11}x_1 +a_{12}x_2 +...+ a_{1n}x_n =b_1\\
a_{21}x_1 +a_{22}x_2 +...+ a_{2n}x_n =b_2\\
a_{ij} = $ are coefficients\\ $
x_i = $ variables = unknowns\\ $
b_i =$ constants\\ 
The set $S_1, ... S_n$ is a solution if  $a_iS_1 +...+a_{in}S_n=b_i$ is True = consistent for all $i$
\\ \textbf{3 possible outcomes}\\
One Solution \\ No Solutions \\ Infinitely many Solutions
\\ \textbf{Row Operations}
\\Interchange Rows \\ Row Multiplication \\ Row Addition: Add a constant multiple of one row to another row
\[
\left[\begin{array}{cc|c}
1 & 0 & S_1 \\
0 & 1 & S_2
\end{array}\right]
\]
\subsection{Row Reductions and Echelon Form}
\subsection{Vector Equations}
\subsection{The Matrix Equations Ax=b}
\subsection{Solution Sets for Linear Systems}
\subsection{Linear Independence}
\subsection{Linear Transformations}
\subsection{The Matrix of a Linear Transformation}
\section{Chapter 2}
\subsection{Matrix Operations}
\subsection{The Inverse of a Matrix}
\subsection{Characterizations of Invertible Matrices}
\section{Chapter 3}
\subsection{Introduction to Determinants}
\subsection{Properties of Determinants}
\subsection{Cramer's Rule and Linear Transformations}
\section{Chapter 4}
\subsection{Vector Spaces and Subspaces}
\subsection{Null Spaces and Column Spaces}
\subsection{Linear Independence}
\subsection{Coordinate Systems}
\subsection{Dimension and Rank}
\subsection{Change of Basis}
\section{Chapter 5}
\subsection{Eigenvectors and Eigenvalues}
\subsection{The Characteristic Equations}
\subsection{Diagonalization}
\subsection{Eigenvectors and Linear Transformations}
\subsection{Complex Eigenvalues}
\subsection{Discrete Dynamical Systems}
\subsection{Applications to Markov Chains}
\section{Chapter 6}
\subsection{Inner Product Spaces}
\subsection{Orthogonal Sets}
\subsection{Orthogonal Projections}
\subsection{The Gram Schmdit Process}
\subsection{Inner Product Spaces}




\pagebreak
\section{Example Problems with Solutions}


\end{document}