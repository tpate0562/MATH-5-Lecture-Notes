 %!TeX root = PHYS42 Lecture Notes.tex
\documentclass{article}
\usepackage[dvipsnames, svgnames, x11names]{xcolor}
\usepackage{tikz}
\usepackage{pgfplots}
\usepackage{pgfplotstable}
\usepackage{setspace}
\usepackage{units}
\usepackage{booktabs}
\usepackage{graphicx}
\usepackage{multirow}
\usepackage{amsopn}
\usepackage{bbding}
\usepackage{amsmath}
\usepackage{hyperref}
\usepackage{cancel}
\usepackage{gensymb}
\usepackage[margin = 1.2in]{geometry}
\AtBeginEnvironment{document}{\everymath{\displaystyle}}
\title{MATH 2 Lecture Notes}
\date{Tuesday, 14 January, 2025}
\author{Tejas Patel}
\begin{document}
\maketitle
\tableofcontents
\pagebreak
\section{Chapter 1}
\subsection{Terminology}
\textbf{Definition} A differential equation is an equation containing the derivatives or differentials 
of one or more dependent variables, with respect to one or more independent variables.\\
\textbf{$\cdot$} An Ordinary Differential Equation (ODE) involves only ordinary derivatives\\
\textbf{$\cdot$} A Partial Differential Equation (PDE) involves partial derivatives.\\
\textbf{Definition} The order of a DE is the order of the highest-order derivative that appears in the DE
\textbf{Notation} $F(x,y,\frac{dy}{dx}, \frac{d^2y}{dx^2})$\\
\textbf{Definition} A linear DE is any DE that can be written in form:\\
${\displaystyle a_{0}(x)y+a_{1}(x)y'+a_{2}(x)y''\cdots +a_{n}(x)y^{(n)}=b(x)}$\\
For a DE to be linear:
\begin{enumerate}
    \item Y and all of its derivatives much be of the 1st degree
    \item Any term that does not include y or any of its derivatives must be a function of x
\end{enumerate}
\textbf{Definition} A \textbf{solution} of a DE is any function defined on some interval \textit{I} which, when substituted
into the DE, reduces it to an \textit{identity} on \textit{I}.

\pagebreak
\section{Example Problems with Solutions}
\subsection{}
Find the value(s) of $m$ such that $y=e^{mx}$ is a solution of $y''+4y'-21y=0$\\
$y'=me^{mx}$, $y''=m^2e^{mx}$\\
$m^2e^{mx}+4me^{mx}{-21e^{mx}=0}$\\
$e^{mx}(m^2+4m-21) = 0$\\ 
$(m^2+4m-21) = 0$ \\ 
\boxed{m=-7, m=3} are both solutions

\end{document}